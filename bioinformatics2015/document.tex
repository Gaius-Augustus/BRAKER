\documentclass{bioinfo}
\usepackage{hyperref}
\copyrightyear{2015}
\pubyear{2015}

\begin{document}
\firstpage{1}

\title[BRAKER1]{BRAKER1: Unsupervised RNA-Seq-Based Genome Annotation with GeneMark-ET and AUGUSTUS}
\author[Hoff \textit{et~al}]{Katharina J. Hoff\,$^{1}$\footnote{to whom correspondence should be addressed}, Simone Lange\,$^{1}$, Alexandre Lomsadze\,$^{2}$, Mark Borodovsky\,$^{2,3,4,5}$ and Mario Stanke\,$^1$}
\address{$^{1}$Ernst Moritz Arndt Universit\"{a}t Greifswald, Institute for Mathematics and Computer Science, Walther-Rathenau-Stra\ss{}e 47, 17487 Greifswald, Germany\\
$^{2}$School of Computational Science and Engineering\\
$^{3}$Center for Bioinformatics and Computational Genomics, Georgia Institute of Technology, Atlanta, GA 30332, USA\\
$^{4}$Department of Biological and Medical Physics, Moscow Institute of Physics and Technology, Dolgoprudny, Moscow Region, Russia\\
$^{5}$Joint Georgia Tech and Emory University Wallace H Coulter Department of Biomedical Engineering, Atlanta, GA 30332, USA}

\history{Received on XXXXX; revised on XXXXX; accepted on XXXXX}

\editor{Associate Editor: XXXXXXX}

\maketitle

\begin{abstract}

\section{Motivation:}
Many genome sequencing projects are accompanied by transcriptome sequencing. GeneMark-ET is a gene prediction tool that incorporates unassembled RNA-Seq reads into unsupervised training and subsequently generates \textit{ab initio} gene predictions. AUGUSTUS is a gene finder that usually requires supervised training and uses information form unassembled RNA-Seq reads in the prediction step. 

\section{Results:}
We present BRAKER1, a pipeline for unsupervised RNA-Seq-based genome annotation that combines the advantages of GeneMark-ET and AUGUSTUS. BRAKER1 requires an RNA-Seq read alignment file and a genome file as input. First, GeneMark-ET performs iternative training and generates initial gene structures. Second, AUGUSTUS uses predicted genes for training and then integrates RNA-Seq read information into final gene predictions. In our experiments, we observed that BRAKER1 was more accurate than MAKER2 when it is using RNA-Seq as sole source for training and prediction. BRAKER1 does not require pre-trained parameters or a separate training step.

\section{Availability:}
BRAKER1 is available for download at \url{http://bioinf.uni-greifswald.de/downloads/} and \url{http://exon.gatech.edu/.}.

\section{Contact:} \href{katharina.hoff@uni-greifswald.de}{katharina.hoff@uni-greifswald.de}
\end{abstract}

\section{Introduction}



\begin{methods}
\section{Pipeline Description}

BRAKER1 is implemented in Perl and requires two input files: an RNA-Seq alignment file in \texttt{bam}-format, and a corresponding genome file in \texttt{fasta}-format. Spliced alignment information is extracted from the RNA-Seq file and stored in \texttt{gff}-format. GeneMark-ET uses the genome file and  the spliced alignment \texttt{gff}-file for RNA-Seq supported unsupervised training. After training, GeneMark-ET creates an \textit{ab initio} gene set. Those gene structures that have support by RNA-Seq alignments in all introns are selected for automated training of AUGUSTUS. After training, AUGUSTUS predicts genes in the intput genome file using spliced alignment information from RNA-Seq as extrinsic evidence. The pipeline is illustrated in figure \ref{pipeline}.

\begin{figure}[!tpb]%figure1
\centerline{\includegraphics[width=\linewidth]{figs/braker_graph2.png}}
\caption{Schematic view of the BRAKER1 pipeline.}\label{pipeline}
\end{figure}


\section{Test Data}


% \begin{table}[!t]
% \processtable{This is table caption\label{Tab:01}}
% {\begin{tabular}{llll}\toprule
% head1 & head2 & head3 & head4\\\midrule
% row1 & row1 & row1 & row1\\
% row2 & row2 & row2 & row2\\
% row3 & row3 & row3 & row3\\
% row4 & row4 & row4 & row4\\\botrule
% \end{tabular}}{This is a footnote}
% \end{table}

\end{methods}


% 
% \begin{figure}[!tpb]%figure2
% %\centerline{\includegraphics{fig02.eps}}
% \caption{Caption, caption.}\label{fig:02}
% \end{figure}

\section{Accuracy Results}



%%%%%%%%%%%%%%%%%%%%%%%%%%%%%%%%%%%%%%%%%%%%%%%%%%%%%%%%%%%%%%%%%%%%%%%%%%%%%%%%%%%%%
%
%     please remove the " % " symbol from \centerline{\includegraphics{fig01.eps}}
%     as it may ignore the figures.
%
%%%%%%%%%%%%%%%%%%%%%%%%%%%%%%%%%%%%%%%%%%%%%%%%%%%%%%%%%%%%%%%%%%%%%%%%%%%%%%%%%%%%%%






\section{Conclusion}


\section*{Acknowledgement}

We would like to thank Mark Yandell and Carson Holt for valuable advice on running MAKER2.

\paragraph{Funding\textcolon} This work is supported by the US National Institutes of Health grant HG000783.

%\bibliographystyle{natbib}
%\bibliographystyle{achemnat}
%\bibliographystyle{plainnat}
%\bibliographystyle{abbrv}
%\bibliographystyle{bioinformatics}
%
%\bibliographystyle{plain}
%
%\bibliography{Document}


\begin{thebibliography}{}
\bibitem[Steijger {\it et~al}., 2013]{RGASP} Steijger,T. and Abril,J.F. and Engstr\"{o}m,P.G. and Kokocinski,F. and The RGASP Consortium, Hubard,T.J. and Guigo,R. and Harrow, J. and Bertone, P. (2013) Assessment of transcript reconstruction methods for
 RNA-seq, {\it Nature Methods}, doi:10.1038/nmeth.271.

\bibitem[Lomsadze {\it et~al}., 2014]{GeneMark-ET} Lomsadze, A. and Burns, P.D. and Borodovsky, M. (2014) Integration of mapped RNA-Seq reads into automatic training of eukaryotic gene finding algorithm, {\it Nucleic Acids Research}, doi:10.1093/nar/gku557.

\bibitem[Stanke \textit{et~al}., 2008]{AUGUSTUS}
Stanke, M. and Diekhans, M. and Baertsch, R. and Haussler, D. (2008) Using native and syntenically mapped cDNA alignments to improve de novo gene finding, \textit{Bioinformatics}, \textbf{24}(5), 637.


\end{thebibliography}
\end{document}
